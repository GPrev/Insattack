\subsubsection{Principe du jeu}
INSAttack est un jeu à deux joueurs au tour par tour. Chaque joueur contrôle un département de l'INSA, parmi : INFO, EII, SRC, SGM, GMA, GC. En fonction de son choix, ses unités disposeront de différentes capacités (voir partie \ref{sec:departements}). Au début du jeu, les unités de chaque joueur sont rassemblées sur sa case de départ. Afin de gagner la partie un joueur doit éliminer toutes les unités adverses ou avoir un maximum de points à la fin du nombre de tours imparti. Chaque case occupée par une unité d'un joueur donné rapporte 2 points à ce joueur. Les unités de certains départements peuvent aussi gagner des points grâce à leurs capacités, ou gagner plus ou moins de points selon les types de cases contrôlées (voir partie \ref{sec:departements}).

\subsubsection{Les déplacements}
Cliquer sur une case du plateau permet d'aficher la liste des unités s'y trouvant. On peut ensuite sélectionner l'une de ces unités en cliquant dessus. Une fois cela fait il devient possible de déplacer l'unité sélectionnée par un clic du bouton droit droit sur la case voulue. L'unité se déplacera si le mouvement est possible, c'est à dire si elle possède suffisament de points de déplacement et si la case lui est adjacente.
Pour aider les débutants, une suggestion de déplacements est activable dans les options.

\subsubsection{Les combats}
Lorsqu'une unité tente de se déplacer sur une case occupée par un département ennemi, un combat a lieu. Pendant le combat, un nombre de tour sera choisi aléatoirement ; durant chaque tour un vainqueur sera calculé en fonction de l'attaque et de la défense des deux unités. L'unité qui perd un tour de combat perd 1 point de vie. Le combat se termine si l'une des unités meurt ou si tous les tours ont été joués. Si le défenseur est mort, l'attaquant se déplace sur la case qu'elle occupait. Quelle que soit l'issue d'un combat, il consomme autant de points de déplacement qu'un mouvement classique.