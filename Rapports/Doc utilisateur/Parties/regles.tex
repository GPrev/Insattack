\subsubsection{Principe du jeu}
INSAttack est un jeu à deux joueurs au tour par tour, chaque joueur choisit son département parmi : INFO, EII, SRC, SGm, GMA, GC. En fonction de son choix ses unités disposeront de différentes capacités\ref{sec:departements}. Au début du jeu, les unités de chaque joueur sont rassemblé sur sa case de départ. Afin de gagner la partie un joueur doit gagner un maximum de points à la fin du nombre de tours imparti. Pour cela chaque case qu'il posséde une lui rapporte 1 point. Certains départements peuvent aussi gagner des points grâce à leurs capacités. L'autre moyen de victoire est l'élimination de toutes le sunités adversaires.

\subsubsection{Les combats}
Il ets bien sûr possible, afin de réduire les points de l'adversaire d'attaquer ses unités pour les détruire. Chaque combat comporte plusieurs dont le nombre est choisi aléatoirement. Lors de chacun de ces tours le rapport de force entre l'atatquant et le défenseur (c'est à dire l'atatque de l'attaquant et la défense de l'atatqué) détermine le pourcentage de chance de gagner de l'attaquant.Attaquer une unité adverse coûte autant de points de mouvement que se déplacer sur la case où elle se trouve. Par ailleurs on ne peut attaquer une unité que si elle se trouve sur une case adjacente. 