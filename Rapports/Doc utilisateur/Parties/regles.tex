\subsubsection{Principe du jeu}
INSAttack est un jeu à deux joueurs au tour par tour, chaque joueur choisit son département parmi : INFO, EII, SRC, SGM, GMA, GC. En fonction de son choix ses unités disposeront de différentes capacités (voir partie \ref{sec:departements}). Au début du jeu, les unités de chaque joueur sont rassemblées sur sa case de départ. Afin de gagner la partie un joueur doit éliminer toutes les unités adverses ou avoir un maximum de points à la fin du nombre de tours imparti. Pour cela chaque case qu'il contrôle, en possedant une unité dessus, lui rapporte 2 points. Certains départements peuvent aussi gagner des points grâce à leurs capacités, ou gagner plus ou moins de points selon les types de cases contrôlées.

\subsubsection{Les déplacements}
Pour déplacer une unité, il faut faire un clic gauche sur la case la contenant, puis sur l'unité voulue dans la liste qui s'affiche. Une fois cela fait vous pouvez la déplacer sur une case en faisant un clic droit dessus. L'unité se déplacera si le mouvement est possible, c'est à dire si elle posséde suffisament de points de déplacement et si la case est adjacente.
Pour aider les débutants, une suggestion de déplacements est activable dans les options.

\subsubsection{Les combats}
Vous pouvez bien sûr attaquer les unités adverses, pour cela déplacez simplement l'une de vos unités sur la case où se trouve votre cible. Le combat sera alors déclenché, un nombre de tour sera choisi aléatoirement, durant chacun d'entre eux les chances de gagner de l'attaquant seront calculées en fonction de son attaque et de la défense de son adversaire. L'unité qui perd un tour de combat perd 1 point de vie. Le combat se termine si l'une des unités meurt ou si tous les tours ont été réalisés. Si l'unité ciblée est morte l'attaquant se déplace sur la case qu'elle occupait. Quelle soit réussie ou non, une attaque consomme autant de points de déplacement qu'un mouvement vers cette case.