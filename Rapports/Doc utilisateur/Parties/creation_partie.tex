Pour créer une nouvelle partie, apputez sur le bouton \emph{Nouvelle partie} au lancement du jeu ou allez dans le menu \emph{Fichier} -> \emph{Nouvelle partie} (attention, tous les changements non sauveardés sur votre partie en cours seront alors perdus).
Ensuite, une boîte de dialogui s'affiche, vous demandant de choisir les options de la partie :

\begin{description}
	\item[Carte jouée :] le type de carte désiré, choisi parmi les suivants :
	\begin{description}
		\item[Carte normale :] une carte de taille moyenne.
		\item[Petite carte :] une petite carte.
		\item[Carte de démonstration :] une carte minuscule.
	\end{description}

	\item[Nom du joueur :] le nom de chaque joueur. Les noms ne peuvent être identiques entre eux.

	\item[Département du joueur :] le département que chaque joueur souhaite incarner. Il est possible pour deux joueur de choisir le même département.
\end{description}

Une fois ces options choisies, cliquez sur \emph{Lancer} pour lancer la partie, ou cliquez sur \emph{Annuler} pour annuler la partie.