Le jeu que nous allons implémenter est tiré de Small World, et les règles différeront légèrement de celles proposées. Les modifications apportées aux règles seront détaillées dans les parties \ref{sec:peuples} et \ref{sec:terrains}.
C'est un jeu se déroulant au tour par tour, où chaque joueur joue un peuple différent. Au début de la partie, chaque joueur dispose de toutes ses unités rassemblées sur une même case, quelque peu éloignée de celles des autres joueurs.
Le but de jeu est d'obtenir le plus de points possible, points qui sont gagnés par la maîtrise des cases du jeu. Les cases valent un point, plus un éventuel bonus. Un joueur peut prendre possession d'une case en plaçant une ou plusieurs unités sur celle-ci.

En déplaçant ses unités sur une case où une ou plusieurs unités d'un autre joueur sont présentes, un combat aura lieu. En cas de victoire l'adversaire ne gagnera plus de points sur cette case.
En fonction de la case sur laquelle elle se trouve, une unité pourra bénéficier de différents bonus ou malus, comme un coût de déplacement réduit ou augmenté, une augmentation ou diminution de son attaque ou sa défense, ou encore un gain de points supplémentaire.