%this file is the first report
%a % comment anything after % until the end of the line

%minimum references to begin our article
\documentclass[12pt]{article}
\usepackage[frenchb]{babel}
\usepackage[utf8]{inputenc}
\usepackage[T1]{fontenc}
\usepackage{graphicx}
\usepackage{fancyhdr}
\usepackage{hyperref}
\usepackage{float}
\usepackage{amsmath}
\usepackage[margin=1in]{geometry}
\usepackage{indentfirst}


\pagestyle{fancy}
%\cfoot{Insattack : Projet de POO}
% the last extension makes it possible to add images

%presentation of the document
\title{Insattack : Projet de POO\smallbreak Rapport de conception}
\author{Baptiste \textsc{Bignon}, Gabriel \textsc{Prevosto}}
\date{12/11/2014}
\setlength\parindent{15pt}
\begin{document}

\maketitle

\begin{figure}[!h]
\centering
\includegraphics[width=\textwidth]{Parties/Images/Logo}
\label{fig:logo}
\end{figure}

\newpage

%to add a table of contents
\tableofcontents
\renewcommand{\contentsname}{Sommaire}
\newpage


\section{Introduction}			\label{sec:introduction}
\newpage

\section{Le jeu}				\label{sec:jeu}
\subsection{Les règles}			\label{sec:regles}
\subsection{Les peuples}			\label{peuples}
\subsection{Les terrains}			\label{terrains}
\newpage

\section{Architecture}			\label{sec:archi}
\subsection{Architecture principale}	\label{sec:architecturePrincipale}
\subsection{Création d'une partie}	\label{sec:creationPartie}
\subsection{Déroulement d'une partie}	\label{sec:deroulementPartie}
\subsection{Cycle de vie des unités}	\label{sec:cycleVieUnites}
\newpage

\section{Implémentation}			\label{sec:implementation}
\subsection{Diagramme de classes}	\label{sec:diagrammeClasses}		\input{"Parties/DiagrammeClasses.tex"}
\subsection{Patrons de Conception}	\label{sec:patronsConception}		\input{"Parties/PatronsConception.tex"}
\newpage

\section{Conclusion} 			\label{sec:conclusion}			La conception du modèle pour le jeu \emph{Insattack} est finie, son implémentation peut donc commencer.
L'interface utilisateur n'a pas encore été pensée ; cependant, le modèle n'y fait pas référence et peut donc être développé avant.
Par la suite, l'interface graphique utilisera le modèle pour gérer la partie.

Les règles du jeu ont été légèrement modifiées afin de correspondre avec l'environnement de l'INSA ; il sera donc peut-être nécessaire de les modifier à nouveau dans un souci d'équilibrage.
Cependant, ces modifications seront simples à effectuer grâce à l'utilisation des patrons de  conception décrits dans la partie \ref{sec:patronsConception}.


\end{document}
